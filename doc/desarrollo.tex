
Se compararán los filtros implementadas en lenguage C compilada con máximas optimizaciones (o3) y la versión ASM utilizando el set de instrucciones
SSE. \\
Para medir el rendimiento usaremos la cantidad de ciclos de ejecución, en particular se tomara la mediana de las mediciones.\\
Se útilizara el registro Time Stamp Counter(TSC) que es global del procesador y se ve afectado por una serie de factores,
como por ejemplo el scheluder para realizar un cambio de contexto, esto implicará contar muchos más ciclos (outliers)
 que si nuestra función se ejecutara sin interrupciones, con lo cual se decide tomar la mediana que es robusta a outliers.\\
El procesamiento digital de imágenes es el conjunto de técnicas que se aplican a las imágenes digitales con el objetivo de mejorar la calidad o facilitar la búsqueda de información. 
\\
Siempre se hacen las cuentas independientemente del contenido de la imagen en si.\\

\subsubsection{edgeSobel}
Tiene cuatro diferentes tipos de iteraci\'on y un caso que es tratado de forma diferente.\\
caso 1:\\
	Llenado de la matriz con los pixeles procesados, se excluyen los bordes pues no se pueden aplicar los operadores.\\
caso 2:\\
	Llenado de los bordes de la columna 1 y n, se utiliza shift para poner los 2 bits en 0.\\
caso 3:\\
	Llenado de la primer y última fila, se limpia un registro xmm (pxor con el mismo) y se pisa la fila 1 y la fila n con ceros.\\
caso 4:\\
	El ultimo caso es para poder llenar la primer fila y la ultima con ceros, la forma de iterar es la misma pero cambia el lugar desde donde inicia cada una.\\

Para la primer forma de iterar la matriz, se tomara la parte baja de un registro xmm y los 8 bytes de la parte baja serán extendidos a 8 words para que al hacer las cuentas no perdamos precisión, como se calcularán 4 pixeles, estaremos trabajando con 6 bytes, los 10 bytes restantes no se utilizarán. Esto solo sirve para llenar la parte de la matriz, por que al querer segir terandom llegaremos a la ultima posicion y necesitariamos leer 10 butes demas, pero nos salfriamos de la matriz.\\

La segunda forma de iterar:\\ responde al problema que surgio anteriormente, entonces ahora al leer en un registro xmm tranjaremos con la parte alta, como necesitamos procesar 4 pixeles a la hora de acceder a la matriz necesitaremos estar 10 bytes (posiciones) atras, asi poder acceder a los 6 bytes con los que se procesan 4, luego por la forma en como iniciamos y dado que la cantidad de pixeles en la fila es muliplo de 8, no podremos  procesar los dos ultimos bytes y estos se haran como caso aparte.\\

La tercer forma de iterar:\\ la matriz es accediendo al ultimo bute de la primer fila, copiar su valor en un xmm, hacer un shift de 2 nyrd para pomer os dos primeros butes en 0 y liegp em shift de 3 butes para reacomodar. com esto se puso en 0 los dos bute de los costados, lego se incementa en cantidad de columnas a la posicion y se procede de la misma forma hasta llegar a la ante-ultima fila.\\

La cuarta forma de iterar:\\
	Primer fila, primero se lee con el registro xmm y se shiftean 8 bytes, se divide la cantidad de columnas por 8 y se procede a iterar cantidad de (cantidad de pixeles en fila/8)-1 para poder llenas la primer fila.\\
	Para la ultima fila, se lee desde 8 bytes antes y se hace un shift para poder poner en 0 los primeros 8 bytes de la ultima fila, luego se procede a iterar cantidad de (cantidad de pixeles en fila/8)-1 para poder llenar la matriz.\\


\subsubsection{Tres Colores}
Descripción de una iteración:\\
Leer de memoria:\\
$Xmm0 = [argb| argb| argb | argb].$\\
Se hacen 3 copias y se las shiftean para tener cada componente en dword:\\ 
$Xmm1 = [r| r| r| r]$\\
$Xmm2 = [g| g| g| g]$\\
$Xmm3 = [b| b| b| b]$\\
Se obtiene la suma en dword:\\
$Xmm1 = [r+g+b| r+g+b| r+g+b| r+g+b]$\\
Se castean a float single precisión, se divide por 3 y se castea a int:\\
$Xmm1 = [0,0,0,W| 0,0,0,W| 0,0,0,W| 0,0,0,W]$(estan en dword, pero ocupan 1 byte)\\
Se acomoda $Xmm1$ para que tenga los 3 bytes de cada dword cargado con $W$:\\
$Xmm1 = [0,W,W,W| 0,W,W,W| 0,W,W,W| 0,W,W,W]$\\
Se acomoda en $Xmm2$ los $W$ de los dos primeros píxeles y en $Xmm3$ los $W$ de los dos segundos píxeles:\\
$Xmm2 = [0,0,0,W|0,W,0,W|0,W,0,W|0,W,0,W]$\\
$Xmm3 = [0,0,0,W|0,W,0,W|0,W,0,W|0,W,0,W]$\\
Se limpa $Xmm0$ para que sea acumulador:\\
$Xmm0 = [0|0|0|0]$\\
Se procesan los crema y con un pand se filtra los que dieron negativo:\\
$Xmm0 = [cremas|cremas|cremas|cremas]$\\
Se procesan los verdes y con un pand se filtra los que dieron negativo:\\
$Xmm0 = [cremas y verdes|cremas y verdes|cremas y verdes|cremas y verdes]$\\
Se procesan los rojos y con un pand se filtra los que dieron negativo:\\
$Xmm0 = [rojos, cremas y verdes|rojos, cremas y verdes|rojos, cremas y verdes|rojos, cremas y verdes]$\\
Se escribe a memoria.\\
\subsubsection{Efecto Bayer}
\subsubsection{Cambia Color}