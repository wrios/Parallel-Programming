Este trabajo practico consiste en implementar filtros gráficos utilizando el modelo de procesamiento SIMD. 
El mismo consta en dos componentes igualmente importantes. En primer lugar aplicaremos lo estudiado en clase programando de manera vectorizada un conjunto de filtros.\\
Se realizarán 4 filtros:\\
\begin{itemize}
    \item Tres Colores
    \item Efecto Bayer
    \item Cambia Color
    \item Edge Sobel
\end{itemize}
El filtro Tres Colores es usado para ...\\
El filtro Bayer es usado para ... . La razón de que se use mayor cantidad de puntos verdes es que el ojo humano es más sencible a ese color.\\
EL filtro Cambia Colores es usado para ...\\
El cuarto filtro es Edge Sobel, el cual utiliza la vecindad de Moore para hacer una aproximaci\'on de las derivadas parciles, este filtro es \'util para poder detectar los bordes de la imagen.\\
